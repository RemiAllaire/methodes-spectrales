\documentclass[11pt]{article}

% Encodage & langue
\usepackage[utf8]{inputenc}
\usepackage[T1]{fontenc}
\usepackage[french]{babel}
\usepackage{lmodern}

% Mise en page
\usepackage[margin=3cm]{geometry}
\usepackage{indentfirst}

% Maths
\usepackage{amsmath, amssymb, amsthm, mathtools, stmaryrd}

% Graphiques & figures
\usepackage{graphicx}
\graphicspath{{./images/}}
\usepackage{subcaption}
\usepackage{float}
\usepackage{diagbox}

% Hyperliens
\usepackage{hyperref}

% Code source
\usepackage{listings}
\usepackage{xcolor}  % requis pour definecolor

\definecolor{dkgreen}{rgb}{0,0.6,0}
\definecolor{gray}{rgb}{0.5,0.5,0.5}
\definecolor{mauve}{rgb}{0.58,0,0.82}

\lstset{
  language=C++, % attention à ne pas doubler les virgules
  frame=tb,
  aboveskip=3mm,
  belowskip=3mm,
  showstringspaces=false,
  columns=flexible,
  basicstyle={\small\ttfamily},
  numbers=none,
  numberstyle=\tiny\color{gray},
  keywordstyle=\color{blue},
  commentstyle=\color{dkgreen},
  stringstyle=\color{mauve},
  breaklines=true,
  breakatwhitespace=true,
  tabsize=3,
  literate=
    {é}{{\'e}}1 {ë}{{\"{e}}}1 {è}{{\`e}}1 {à}{{\`a}}1 {ç}{{\c{c}}}1
    {œ}{{\oe}}1 {ù}{{\`u}}1 {É}{{\'E}}1 {È}{{\`E}}1 {À}{{\`A}}1 {Ç}{{\c{C}}}1
    {Œ}{{\OE}}1 {Ê}{{\^E}}1 {ê}{{\^e}}1 {î}{{\^i}}1 {ô}{{\^o}}1
    {û}{{\^u}}1 {â}{{\^a}}1 {Â}{{\^A}}1 {Î}{{\^I}}1
}

% Listes
\usepackage[shortlabels]{enumitem}

% Théorèmes
\newtheorem{theorem}{Théorème}[section]
\newtheorem{lemma}[theorem]{Lemme}
\newtheorem{corollary}[theorem]{Corollaire}

% Table des matières
\usepackage{tocloft}

% En-têtes et pieds de page
\usepackage{fancyhdr}
\pagestyle{fancy}
\fancyhf{}
\rhead{\thepage}
\lhead{}
\cfoot{}

% Commentaires en bloc
\usepackage{comment}

\title{title}
\author{Remi}
\date{2025}

\begin{document}

\maketitle

\hypertarget{comment-ruxe9soudre-probluxe8me-de-valeurs-propres-avec-galerkin.}{%
\subsection{Comment résoudre problème de valeurs propres avec
Galerkin.}\label{comment-ruxe9soudre-probluxe8me-de-valeurs-propres-avec-galerkin.}}

Soit \(H\), l'opérateur Hamiltonien, on veut résoudre l'équation de
Schrödinger. \[
H\psi = E\psi
\] Soit \(\phi_{i}\), les fonctions test et de base (orthonormale,
\textasciitilde complete). On cherche une solution \(\psi _N\) de la
forme \[
\psi_{N} = \sum_{n=0}^{N}c_{n}\phi_{n}
\] On souhaite minimiser l'équation résiduelle \[
R(x) = H\psi_{N}-E\psi_{N}
\] Le développement spectral de l'équation résiduelle \[
R =\sum_{n=0}^{N}r_{n}\phi_{n}
\] Par orthogonalité, on obtient \[
r_{n} =(\phi_{n},R)
\] En utilisant la méthode de Galerkin, on imposant
\(r_{n}= 0, \quad n= 0,\dots,N\), on peut simplifier \[
(\phi_{i},H\phi_{j}-E\phi_{j})(a_{i}) = (0)
\] Cela donne une équation homogène. On cherche les solutions
non-triviales afin de calculer les niveaux d'énergie (valeurs propres).
En posant, \[
A_{ij} =(\phi_{i},H\phi_{j}-E\phi_{j}) 
\] on cherche E tel que \[
\det(A) = 0
\]

\hypertarget{thuxe9orie-uxe9quation-de-schruxf6dinger-en-1d}{%
\subsection{Théorie équation de Schrödinger en
1D}\label{thuxe9orie-uxe9quation-de-schruxf6dinger-en-1d}}

L'équation de Schrödinger en 1 dimension

\[
H\psi=E\psi
\] où \(H\) est le Hamiltonien défini avec un potentiel \(V(x)\) comme

\[
\begin{aligned}
H = -\frac{1}{2} \frac{d^{2}}{dx^{2}} +V(x)
\end{aligned}
\]

On étudie dans notre cas, on étudie l'oscillateur harmonique quantique,
ainsi

\[
H = -\frac{1}{2} \frac{d^{2}}{dx^{2}} + \frac{1}{2}x^{2}
\]

Les états propres, à paramètre près sont

\[
E_{n}=n+\frac{1}{2}
\]

\hypertarget{fonctions-propres}{%
\subsubsection{Fonctions propres}\label{fonctions-propres}}

Les fonctions propres sont les fonctions d'Hermite généralisés.

\[
\psi_{n}(x)= \frac{e^{x^{2}/2}}{\sqrt{2^{n}n!\sqrt{ \pi }  }}H_{n}(x)
\]

Travaillons avec une légère modification comme fonction de base

\[
\varphi_{n}(x)= \frac{e^{-(\beta x)^{2}/2}}{\sqrt{\beta2^{n}n!\sqrt{ \pi }  }}H_{n}(\beta x)
\]

On peut exploiter la propriété des polynômes d'Hermite

\[
\int_{-\infty}^{\infty} H_{n}H_{m} e^{-x^{2}} \, dx = 2^{n}n!\sqrt{ \pi }\delta_{nm} 
\]

\[
\int_{-\infty}^{\infty} \varphi_{n}\varphi_{m} \, dx= \delta_{nm} 
\]

\hypertarget{calcul-analytique}{%
\subsubsection{Calcul analytique}\label{calcul-analytique}}

\hypertarget{effet-du-hamiltonien}{%
\paragraph{Effet du hamiltonien}\label{effet-du-hamiltonien}}

On veut calculer \[
\begin{aligned}
(\varphi_{i},H\varphi_{j}-E\varphi_{j})  & =(\varphi_{i},H\varphi_{j})-(\varphi_{i},E\varphi_{j}) \\
 & =(\varphi_{i},H\varphi_{j})-E\delta _{ij}
\end{aligned}
\]

Commençons par les dérivées de \(\varphi_{i}\). On pose
\(N_{i}=\frac{1}{\sqrt{ \beta_{2}^{i}i!\sqrt{ \pi } }}\) et
\(z = \beta x\)

\[
\begin{aligned}
(N_{i}e^{-(z)^{2}}H_{i}(z))' & = N_{i}(-\beta ze^{-z^{2}/2}H_{i}(z)+\beta e^{-z^{2}/2}H'_{i}(z)) \\
 & =N_{i}\beta e^{-z^{2}/2}(H'_{i}(z)-zH_{i}(z)) \\
\left[N_{i}e^{-z^{2}}H_{i}(z)\right]'' & = \beta^{2}N_{i}​e^{−z^{2}/2}\left[H_{i}''​(z)−2zH_{i}'​(z)+(z^{2}−1)H_{i}​(z)\right]
\end{aligned}
\]

Avec l'identité

\[
H''_{n}(z)=2zH_{n}'(z)-2nH_{n}(z)
\]

On abouti à

\[
\varphi_{i}''=\beta^{2}N_{i}e^{-z^{2}/2}(z^{2}-2i-1)H_{i}(z)
\]

\[
\varphi_{i}''=\beta^{2}(\beta^{2}x^{2}-2i-1)\varphi_{i}
\]

On a

\[
\begin{aligned}
H\varphi_{j} & = -\frac{1}{2} \frac{d^{2}\varphi_{j}}{dx^{2}} + \frac{1}{2}x^{2}\varphi_{j} \\
&=-\frac{1}{2}\beta^{2}((\beta^{2}-1)x^{2}-2j-1)\varphi_{j} \\
&= \frac{1}{2}((1-\beta^{4})x^{2}+\beta^{2}(1+2j))\varphi_{j}
\end{aligned}
\]

\hypertarget{calcul-de-lintuxe9grale}{%
\paragraph{Calcul de l'intégrale}\label{calcul-de-lintuxe9grale}}

Le calcul analytique du Hamiltonien permet de calculer analytiquement
les intégrales de la méthode de Galerkin. On exploite l'orthonormalité
des fonctions de bases. Cependant, on doit travailler un peu pour le
terme multiplié par \(x^{2}\).

\hypertarget{identituxe9}{%
\paragraph{Identité}\label{identituxe9}}

On a

\[
xH_{n}(x)+ \frac{1}{2}H_{n+1}(x)+nH_{n-1}(x)
\]

On peut aussi l'appliquer aux fonctions d'Hermite (\(\beta =1\)) pour
aboutir à

\[
x\psi_{n}(x)= \sqrt{ \frac{n}{2} }\psi_{n-1}(x)+\sqrt{ \frac{n+1}{2} }\psi_{n+1}(x)
\] Comme les fonctions d'Hermite généralisées peuvent s'écrire comme
\(\varphi_{n}^{\beta}(x) =\frac{1}{\sqrt{ \beta }}\psi(\beta x)\) On a

\[
\begin{align}
(\varphi_{i},H\varphi_{j}) & =\int_{-\infty}^{\infty} \varphi_{i}H\varphi_{j} \, dx  \\
&=\frac{1}{2} (1-\beta^{4})\int_{-\infty}^{\infty} \varphi_{i}\varphi_{j}x^{2} \, dx + \frac{1}{2}\beta^{2}(1+2j)\delta_{ij} 
\end{align}
\]

\end{document}
